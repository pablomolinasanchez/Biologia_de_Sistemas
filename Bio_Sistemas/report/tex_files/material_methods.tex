
\section{Materiales y métodos}

\subsection{Materiales}

En esta sección vamos a explicar los recursos que hemos usado durante nuestra investigación referente a la trombosis arterial.

\subsubsection{Human Phenotype Ontology}

HPO (Human Phenotype Ontology)\cite{HPO} es un vocabulario estandarizado de anomalías fenotípicas en enfermedades humanas que utiliza un fenotipado preciso y detallado para poder ser usado a nivel computacional en distintos campos de la medicina.


\subsubsection{StringDB}

STRING (Search Tool for the Retrieval of Interacting Genes/Proteins) \cite{STRING} es una base de datos biológica sobre interacciones entre proteínas, tanto interacciones conocidas como previstas. Estas interacciones incluyen tanto asociaciones directas como indirectas.

\subsubsection{ClusterProfiler}

Paquete de R que proporciona diferentes funciones para el análisis funcional de genes.

\subsubsection{igraph}

Es una librería diseñada para la investigación en la ciencia de redes. Esta que nos permite crear, manipular y analizar redes y gráficos. Se puede usar con lenguajes de alto nivel como R y Python.

\subsubsection{Linkcomm}

Las comunidades nos dan información sobre la estructura y las conexiones que forman los diferentes nodos de una red, pudiendo así identificar aquellos que forman conexiones con múltiples comunidades. Linkcomm \cite{linkcomm} nos proporciona herramientas para generar y manipular estas comunidades sin importar su tamaño y tipo.



\subsection{Métodos}

En este apartado se explicará los diferentes métodos que se han seguido a lo largo del proceso de análisis de la formación de un trombo arterial investigando las interacciones entre los genes que intervienen en la enfermedad.


\subsubsection{Obtención de proteínas y genes asociados}

El primer paso fue obtener los genes participantes en la trombosis arterial. Para ello primero tuvimos que apoyarnos en el \textit{HPO} de la enfermedad \cite{HPO} para obtener el listado de 25 genes. Posteriormente, nos apoyamos en la web de \textit{STRING} \cite{STRING} , en la que, tras pasar como parámetro el conjunto de genes anterior, obtuvimos una representación gráfica de las interacciones entre los 25 genes. El resultado obtenido fue el mostrado en la figura \ref{fig: Figura 2}. También desde STRING obtuvimos las interacciones entre genes y el grado de cada uno de ellos, los cuales utilizaremos posteriormente para el desarrollo del código. Se comentará más adelante en la sección \textit{Generación de análisis de red PPI}. \\



\subsubsection{Generación y análisis de red PPI}
Una vez obtenido los genes participantes en la trombosis arterial, comenzamos con nuestro análisis de red PPI. Para ello, primeramente hemos creado el grafo con los genes asociados y con las funciones del paquete \textit{igraph} \cite{igraph}. \\

Tras el análisis del grado, hemos llevado a cabo una \textbf{propagación de genes} a través de \textit{STRING} \cite{STRING} con un número máximo de 45 genes semilla nuevos con una confianza media de 0,4. El objetivo de realizar la propagación de genes es mejorar la conectividad de la red, además de evitar que posibles genes se encuentren desconectados de nuestro sistema a estudiar, haciendo que los análisis que realicemos sean más esclarecedores, ya que tenemos una mejor visión del contexto de la enfermedad. \\

Tras esto, hemos utilizado el paquete \textit{linkcomm} \cite{linkcomm} para llevar a cabo un \textbf{análisis de las comunidades del sistema}, pudiendo determinar los genes que conectan comunidades, así como el tamaño de ellas entre otros distintos análisis que explicaremos en profundidad en la sección \textit{Resultados} .\\

Por último, hemos llevado a cabo un \textbf{análisis funcional} mediante el enriquecimiento con GO del paquete \textit{clusterProfiler} \cite{clusterProfiler} de los procesos biológicos en los que intervienen los genes integrantes de cada una de las distintas comunidades anteriormente obtenidas, apoyándonos en el conjunto genómico del \textit{Homo Sapiens Sapiens} del paquete \textit{DOSE} \cite{DOSE}.  Así, hemos obtenido las distintas funciones biológicas que se llevan a cabo en el proceso de la trombosis arterial, además de poder mostrar que genes son partícipes de cada una de las fases de formación del trombo.
