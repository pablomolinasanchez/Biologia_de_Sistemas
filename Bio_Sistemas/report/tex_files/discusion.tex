\section{Discusión}
\hfill


\hfill

Respecto a los genes que unen comunidades, graficados en la figura \ref*{fig: Figura 7} vemos al TP53 y AKT1 como los genes que pertenecen a más comunidades. El gen TP53 es un regula la división celular impidiendo que las células crezcan y se dividan (proliferen) demasiado rápido o de forma incontrolada \cite{TP53}. Por tanto, sabiendo su función, podemos afirmar que tiene un papel regulador en los procesos biológicos de todas y cada una de las comunidades a las que pertenece. Este sería el motivo por el que une tantas comunidades, además de ser el motivo por el que adquiere gran importancia en el reactoma.\\\\  A su vez, el gen AKT1 proporciona instrucciones para fabricar una proteína denominada quinasa AKT1 \cite{AKT1}. Esta proteína se encuentra en varios tipos de células de todo el organismo, donde desempeña un papel fundamental en muchas vías de señalización \cite{AKT1}. Por tanto, este sería el motivo por el que une tantas comunidades y, al igual que TP53, adquiere especial relevancia en el estudio.\\\\
Ambos pertenecen a la \textit{comunidad 16}, comunidad de mayor tamaño del reactoma. Esta comunidad interviene en cualquier proceso que provoque un cambio en el estado del organismo. En nuestro caso, nos referimos al pathway JAK-STAT, nombrado en \textit{Genes Asociados} y mediante el cual obtenemos las células T-helper \cite{pathway}, que provoca que los linfocitos B liberen anticuerpos hacia la placa aterosclerótica, interpretada como cuerpo extraño.\\\\
A raíz de el artículo recomendado \cite{F2_F9} por la base de datos STRING \cite{STRING}  y nuestro resultado del análisis funcional de la \textit{comunidad 1}, hemos podido observar la importancia en la enfermedad de un gen en concreto, el gen \textbf{F2}, el cual es activado por el gen \textbf{F9}. Este gen F2 es el encargado de sintetizar \textit{protrombina}, inofensiva hasta que ocurre un daño, donde se convierte en la \textit{trombina}, su forma activa. Esta trombina convierte el fibrinógeno en fibrina, que juega un papel muy importante en la formación y proceso trombótico como hemos explicado en la sección \textit{Introducción}. También, perteneciente a esta comunidad hemos visto el gen \textbf{SERPINC1}, que sintetiza la t\textit{antitrombina}, un inhibidor de la trombina \cite{SERPINC1}. Una mutación en este gen provocaría un aumento del riesgo de eventos trombóticos. Si por alguna otra causa también muta el gen F2 y aumenta su actividad, tendríamos un entorno muy favorable a la formación de trombos.\\\\
También perteneciente a esta comunidad, tenemos el PROS1, que sirve como cofactor de la proteína C activada de tipo 4, cuyo pathway de formación es iniciado por el gen C4A. Estos dos genes, intervienen en la inactivación proteolítica de factores de coagulación que el cuerpo para su eliminación a través de células inmunitarias. Por tanto, tratan de eliminar el resultado de la sintetización de la protrombina de la F2.\\\\ Esta comunidad interviene en la coagulación a través de mutaciones en el gen F2 y SERPINC1, que provocan la saturación e inoperancia de los genes PROS1 y C4A en su intento por eliminar ese desorden homeostático de fibrina. 


