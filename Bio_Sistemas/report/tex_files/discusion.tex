\section{Discusión}
\hfill


\hfill

Respecto a los genes que unen comunidades, graficados en la figura \ref*{fig: Figura 7} vemos al \textbf{TP53} y \textbf{AKT1} como los genes que pertenecen a más comunidades. El gen TP53 es un regulador la división celular impidiendo que las células crezcan y se dividan (proliferen) demasiado rápido o de forma incontrolada \cite{TP53}. Por tanto, sabiendo su función, podemos afirmar que tiene un papel regulador en los procesos biológicos de todas y cada una de las comunidades a las que pertenece. Este sería el motivo por el que une tantas comunidades, además de ser el motivo por el que adquiere gran importancia en el reactoma.\\\\  A su vez, el gen AKT1 proporciona instrucciones para fabricar una proteína denominada quinasa AKT1 \cite{AKT1}. Esta proteína se encuentra en varios tipos de células de todo el organismo, donde desempeña un papel fundamental en muchas vías de señalización \cite{AKT1}. Por tanto, este sería el motivo por el que une tantas comunidades y, al igual que TP53, adquiere especial relevancia en el estudio.\\\\
Ambos pertenecen a la \textit{comunidad 16}, comunidad de mayor tamaño del reactoma. Esta comunidad interviene en cualquier proceso que provoque un cambio en el estado del organismo. En nuestro caso, nos referimos al pathway \textbf{JAK-STAT}, nombrado en \textit{Genes Asociados} y mediante el cual obtenemos las células T-helper \cite{pathway}, que provoca que los linfocitos B liberen anticuerpos hacia la placa aterosclerótica, interpretada como cuerpo extraño.\\\\
En consonancia nuevamente con la introducción y los resultados funcionales de las comunidades, hemos podido observar que adquiere gran importancia en la enfermedad la \textit{comunidad 1}. A raíz de el artículo recomendado \cite{F2_F9} por la base de datos STRING \cite{STRING}  y nuestro resultado del análisis funcional de la \textit{comunidad 1}, hemos podido observar la importancia en la enfermedad de un gen en concreto, el gen \textbf{F2}, el cual es activado por el gen \textbf{F9}. Este gen F2 es el encargado de sintetizar \textit{protrombina}, inofensiva hasta que ocurre un daño, donde se convierte en la \textit{trombina}, su forma activa. Esta trombina convierte el fibrinógeno en fibrina, que juega un papel muy importante en la formación y proceso trombótico como hemos explicado en la sección \textit{Introducción}.
El gen \textbf{SERPINC1}, sintetiza la \textit{antitrombina}, un inhibidor de la trombina \cite{SERPINC1}, por lo que es el inhibidor negativo del gen F2. \\\\
También perteneciente a esta comunidad, tenemos el \textbf{PROS1}, que sirve como cofactor de la proteína C activada de tipo 4, cuyo pathway de formación es iniciado por el gen \textbf{C4A} \cite{PROS1}. Estos dos genes, intervienen en la inactivación proteolítica de factores de coagulación que el cuerpo para su eliminación a través de células inmunitarias. Por tanto, tratan de eliminar el resultado de la sintetización de la protrombina de la F2. Así, una mutación en el gen F2 provocaría aumento de su actividad y, si por algún causal muta tambien el gen SERPINC1, tendríamos un entorno muy favorable a la formación de trombos o aceleraría la formación del mismo, ya que provocarían la saturación e inoperancia de los genes PROS1 y C4A en su intento por eliminar ese desorden homeostático de fibrina.\\\\
Por último, gracias a nuestros resultados de los análisis funcionales de las comunidades, hemos podido concretar a los responsables de la formación de plaquetas. Esta sería la formada por la \textit{comunidad 15}, comunidad independiente representada en la figura \ref{fig: Figura 5}. Esta comunidad estaría conformada por el gen \textbf{THPO}. Este gen es el encargado de codificar la \textit{trombopoyetina} uniéndose a su receptor\cite{THPO_MPL}, codificado por el gen \textbf{MPL} \cite{THPO_MPL} y regulado negativamente por el gen \textbf{SH2B3} \cite{SH2B3}. Esta unión provoca la activación de la trombopoyetina, que adquiere una gran importancia en la enfermedad puesto que ayuda a producir células sanguíneas, especialmente plaquetas \cite{THPO_MPL}. Cualquier mutación en estos genes generaría un entorno hiperplaquetario y, por tanto, favorable a la aparición o formación de un trombo, por lo que es una comunidad a tener en cuenta en el contexto de la enfermedad.\\\\
Respecto al resto de comunidades no mencionadas, no las consideramos de especial relevancia en nuestro estudio, puesto que algunas comunidades son vías reguladores de procesos cardíacos, comunidades parálogas de algunas secciones de la \textit{comunidad 16} o de creación de eritrocitos, entre otros. Estas comunidades han sido, en su mayoría, genes semillas añadidos para obtener más información de las comunidades que realmente afectan a la trombosis arterial.


